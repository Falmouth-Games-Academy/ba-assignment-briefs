\documentclass{../../fal_assignment}
\graphicspath{ {../../} }

\usepackage{enumitem}
\setlist{nosep} % Make enumerate / itemize lists more closely spaced
\usepackage[T1]{fontenc} % http://tex.stackexchange.com/a/17858
\usepackage{url}
\usepackage{todonotes}

\title{VR Design 1 - Portfolio}
\author{Alcwyn Parker}
\module{VR120}
\version{1.0}

\begin{document}

\maketitle

\section*{Introduction}

\begin{marginquote}
``VR at its very core, virtual reality is about being freed from the limitations of actual reality..''

--- John Carmack

\end{marginquote}
\marginpicture{flavour_pic}{
    Screenshot from: Family, directed by Andrew Thomas Huang, with co-creative direction by Björk and James Merry
}

Virtual reality has the potential to immerse users in new worlds, allowing them to embody new forms and interact with their surroundings in novel ways. With recent advancements in haptic controllers and the proliferation of VR modelling tools, it seems likely that in the future we will see more game artist working in VR to sculpt and model assets for games. Tools such as Quill, Medium and Tilt Brush provide intuitive ways to interface with design tools. These tools priorities expression and creativity over precision and photorealism. This assignment asks you to experiment with the creative potential of VR without getting hung up on technical pipelines and complex tools. For this assignment, you are required to submit a portfolio of experimental VR assets and one final VR artefact: a virtual world to be experienced in VR built using original assets from your portfolio. \textbf{ You are only allowed to create assets whilst in VR.} You may use any VR tools you choose although the tuition will focus on Tilt Brush and corresponding Unity pipelines. 

This assignment is formed of several parts:

\begin{enumerate}[label=(\Alph*)]
    \item \textbf{Conceive}, of a virtual world based on a cultural artefact such as a poem, piece of art or song lyric and then:
    	\begin{enumerate}[label=\roman*.]
    		\item \textbf{Research} the chosen cultural artefact 
    		\item \textbf{Collect} references in the form of mood boards;
    		\item \textbf{Sketch} out the world focusing on the assets required to build it;
	\end{enumerate}
    \item \textbf{Create}, a portfolio of assets and experiments based on the work carried out for part A:
    	\begin{enumerate}[label=\roman*.]
    		\item \textbf{Experiment} with different VR modelling tools and techniques;
    		\item \textbf{Collate} the work in a manner that is easy to navigate such as through SketchFab or as blog content. 
    		\item \textbf{Present}, as an individual, your work in progress and receive feedback from tutors and peers
	\end{enumerate}
    \item \textbf{Construct} a world using the portfolio of assets:
    	\begin{enumerate}[label=\roman*.]
    		\item \textbf{Integrate} the tools and techniques evidenced in assignment 1.
    		\item \textbf{Enhance} the assets using tools inside Unity Game Engine
    		\item \textbf{Design} a sound scape to compliment the world
    		\item \textbf{Consider} adding multi-player. This is a stretch goal and is not required by the assignment
	\end{enumerate}
\end{enumerate}

\subsection*{Assignment Submission}
This assignment is an \textbf{individual} task. You should submit a zip folder of your work that includes a readme.txt  that outlines the submission directory structure and provides \textbf{links to any work that is too large} for the Learning Space size constraint (1GB). Your submission should include sub-folders for:
\begin{enumerate}[label=\roman*.]
    	\item \textbf{Research} - any supporting research relevant to the assignment
    	\item \textbf{Experiments} - any supporting experimentation relevant to the assignment
    	\item \textbf{Design} - any supporting design work relevant to the assignment
    	\item \textbf{Final} - a video of the final experience
\end{enumerate}

\section*{FAQ}

\begin{itemize}
	\item 	\textbf{What is the deadline for this assignment?} \\ 
    		Falmouth University policy states that deadlines must only be specified on the MyFalmouth system.
    		
	\item 	\textbf{What should I do to seek help?} \\ 
    		You can email your tutor for informal clarifications. For informal feedback, make a pull request on GitHub. 
    		
\end{itemize}

\section*{Additional Resources}

%\begin{itemize}
%    \item Guzdial, M.J . and Ericson, B. (2015) Introduction to Computing and Programming in Python: A Multimedia Approach, 4th Edition. Pearson: New York.
%    \item Martin, R.C. (2008) Clean Code: A Handbook of Agile Software Craftsmanship. Prentice Hall: New York
%    \item http://guide.agilealliance.org/guide/pairing.html
%   \item http://www.pairprogramming.co.uk/
%   \item http://www.pythontutor.com/
%\end{itemize}

\rubrichead{All submissions and assessment criteria for this assignment are individual.}
\begin{markingrubric}
%
    \firstcriterion{Basic Competency Threshold}{30\%}
        \gradespan{1}{\fail At least one part is missing or is inadequate.}
        \gradespan{5}{Adequate ability to generate ideas, problem solving, concepts, technical competency and proposals in response to set briefs and/or self-initiated activity.
        	\par The work demonstrates an adequate, ethically informed, real-world experience of industry/business environments and markets.
	\par Enough work is available to hold a meaningful discussion.
	\par Adequate participation in-class peer-review activities at least at the level of basic competency.
	\par Clear evidence of design knowledge.
	\par No breaches of academic integrity.}
%
    \criterion{RESEARCH: \\Depth of Research}{10\%}
        \grade\fail 	No research has been carried put.
            \par 		The research does not inform the design or implementation.
        \grade 		little research has been carried out
            \par 		The research is not evident in the design or implementation
        \grade 		Some research has been carried out
            \par 		There is some evidence that the research has informed the design or implementation
        \grade 		Much research has been carried out
            \par 		There is some evidence that the research has informed the design and implementation
        \grade 		Considerable research has been carried out
            \par 		There is much evidence that the research has informed the design and implementation 
        \grade 		Significant research has been carried out
            \par		There is much evidence that the research has informed every aspect of the design and implementation 
            \par		Design choices have been informed by research 
%
    \criterion{PRACTICE: \\Design Process}{20\%}
        \grade\fail No evidence of a design process
            \par No sketches of mood boards
        \grade Little evidence of a design process
            \par one or two sketches are present
            \par the design work does not correspond with the final experience
        \grade Some evidence of a design process
            \par Sketches and mood boards present
            \par Sketches and mood boards have some correlation with the final artefact
        \grade Much evidence of a design process
            \par Sketches and mood boards directly correlate to the final artefact
        \grade Considerable evidence of a design process
            \par Sketches and mood boards directly correlate to the final artefact
            \par Prototypes feedback into the design process
        \grade Significant evidence of a design process
            \par Sketches and mood boards directly correlate to the final artefact
            \par Prototypes feedback into the design process throughout the entire project
%
    \criterion{PRACTICE: \\Consistency of Assets}{20\%}
        \grade\fail There are no original assets
            \par assets are borrowed from on-line stores
            \par aesthetic style is inconsistent and at odds with the world concept
        \grade Little consistency of visual style in relation to the world concept
            \par A VR Modelling tool has been utilised to create basic assets 
        \grade Some consistency of visual style in relation to the world concept
            \par A VR Modelling tool has been utilised to create assets    
        \grade Much consistency of visual style in relation to the world concept
            \par VR Modelling tools have been utilised to create assets
        \grade Considerable consistency of visual style in relation to the world concept
            \par VR Modelling tools have been utilised to create high quality assets
        \grade Significant consistency of visual style in relation to the world concept
            \par A variety of VR Modelling tools and techniques have been utilised to create high quality assets    
        
%
    \criterion{IMPLEMENT: \\Creative Response to Brief}{10\%}
        \grade\fail No creativity.
            \par The work is a clone of an existing work with mere cosmetic alterations.
        \grade Little creativity.
            \par The work is derivative of existing works, with only minor alterations.
        \grade Some creativity.
            \par The work is derivative of existing works, demonstrating little divergent and/or subversive thinking.
        \grade Much creativity.
            \par The work is somewhat novel, demonstrating some divergent and/or subversive thinking.
        \grade Considerable creativity.
            \par The work is novel, demonstrating significant divergent and/or subversive thinking.
        \grade Significant creativity.
            \par The work is highly original, with strong evidence of divergent and/or subversive thinking.
%
    \criterion{IMPLEMENT: \\Sophistication of World}{10\%}
       \grade\fail No sophistication.
            \par The work is minimal with one or no original assets
        \grade Little sophistication
            \par The work is minimal with only a few original assets
        \grade Some sophistication
            \par The work contains enough original assets to create a novel experience
        \grade Much sophistication
            \par The work contains enough original assets and experimentation in engine to create an engaging and novel experience
        \grade Considerable sophistication
            \par The work contains a wealth of original assets, animations and experimentation in engine. The final experience feels reasonably compelling as a unique and novel virtual world. 
        \grade Significant creativity.
            \par The work contains a vast arry of original assets, animations and experimentation in engine. The final experience feels compelling as a unique and novel virtual world. 
%
    
%
\end{markingrubric}

\end{document}