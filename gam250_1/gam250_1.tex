\documentclass{../fal_assignment}
\graphicspath{ {../} }

\usepackage{enumitem}
\setlist{nosep} % Make enumerate / itemize lists more closely spaced
\usepackage[T1]{fontenc} % http://tex.stackexchange.com/a/17858
\usepackage{url}
\usepackage{todonotes}

\title{Advanced Programming - Assignment 1}
\author{Brian McDonald}
\module{GAM250}

\begin{document}

\maketitle

\section*{Introduction}

\begin{marginquote}
If we knew what it was we were doing, it would not be called research, would it?

--- Albert Einstein

\end{marginquote}

In this assignment you will be developing your research skills in order to expand your programming knowledge. This research will inform the direction of Assignment 2 and even feed into your 2nd year group project.

Research is a key skill in any programmers toolbox and during your career as a Games Developer you will encounter problems that will require a significant amount of research.

This assignment is formed of several parts:

\begin{enumerate}[label=(\Alph*)]
	\item \textbf{Provide} a project proposal, this should contain the following:
	\begin{enumerate}
		\item Project Title
		\item Project Description
		\item Project Justification
		\item List of key sources  (a minimum of 5, including 3 Journals) 
	\end{enumerate}
    \item A \textbf{draft} of your research
    \item A \textbf{fully referenced document} which contains your research
\end{enumerate}

\subsection*{Part A}

Part A is \textbf{single summative submission}. This is \textbf{individual} work will be assessed on a \textbf{threshold} basis. The following criteria are used to determine a pass or fail: 

\begin{enumerate}[label=(\alph*)]
	\item Submission is timely;
	\item Reasonable justification for the project
	\item At least 5 sources included (minimum of 3 Journals)
\end{enumerate}

To complete Part A, you should submit a document to the assignment area of the learning space by 

You will receive informal feedback via email from your \textbf{tutor}.

\subsection*{Part B}

Part B is a \textbf{single summative submission}. This is \textbf{individual} work will be assessed on a \textbf{threshold} basis. The following criteria are used to determine a pass or fail: 

\begin{enumerate}[label=(\alph*)]
	\item Submission is timely;
	\item Varied list of references
	\item A reasonable standard of English
\end{enumerate}

To complete Part B, you should submit a document to the assignment area of the learning space by 

You will receive \textbf{immediate informal} feedback from your \textbf{tutor}.


\subsection*{Part C}

Part C is a \textbf{single formal submission}. This is \textbf{individual} work will be assessed on a \textbf{threshold} basis. The following criteria are used to determine a pass or fail: 

\begin{enumerate}[label=(\alph*)]
	\item Submission is timely;
	\item Varied list of references
	\item A reasonable standard of English
	\item Incorporated feedback from Part B
\end{enumerate}

To complete Part C, you should submit a document to the assignment area of the learning space by 

You will receive \textbf{formal} feedback on the learning space

\section*{Additional Guidance}
Research is one of the key skills you can develop as a programmer, during you career you will encounter problems which you can't solve straight away. This will require you to go away and carry out research to gather the knowledge to give you the tools to solve the problem. In addition to this, research is simply not about parroting what has been written before, you have to critically analyse and then adapt to your own Use Case.

At Falmouth University we use the Harvard Referencing style, please ensure that you use this  for all references. You can receive support for this format from your tutor and the University Library - \url{http://library.fxplus.ac.uk/library/how/referencing}. You are also welcome to use tools such as RefWorks, Mendely, or Zotero to organise your references, these tools will often allow you to export a reading list in a Harvard style which can then be incorporated into your report.

Writing takes practice so make sure that you take feedback from your tutor and incorpate the changes into a new version of the document. You should also consider getting someone else to proof read your report, it is very easy to make mistakes and not notice them when you are too close to the work.

\section*{FAQ}

\begin{itemize}
	\item 	\textbf{What is the deadline for this assignment?} \\ 
    		Falmouth University policy states that deadlines must only be specified on the MyFalmouth system.
    		
	\item 	\textbf{What should I do to seek help?} \\ 
    		You can email your tutor for informal clarifications. For informal feedback, make a pull request on GitHub. 
    		
    	\item 	\textbf{Is this a mistake?} \\ 	
    		If you have discovered an issue with the brief itself, the source files are available at: \\
    		\url{https://github.com/Falmouth-Games-Academy/ba-assignment-briefs}.\\
    		 Please raise an issue and comment accordingly.
\end{itemize}

\section*{Additional Resources}

\begin{itemize}
     \item Elements of Style
     \item Research methods book
     \item Harvard reference
     \item GDC Vault
     \item SIGGRAPH
     \item DiGRA
     \item FDG
     \item International conference in computational creativity
     
    
\end{itemize}

\begin{markingrubric}
    \firstcriterion{Basic Competency Threshold}{40\%}
        \gradespan{1}{\fail At least one part is missing or is unsatisfactory. 
        
        There is little or no evidence of an iterative development process and no improvement over time in regards to the quality of the design and build of the prototype.}
        \gradespan{5}{Submission is timely.
        	\par Enough work is available to hold a meaningful discussion.
	\par Clear evidence of a `reasonable' iterative development process
	\par Clear evidence of programming knowledge and communication skills.
	\par Clear evidence of reflection on own performance and contribution.
	\par Only constructive criticism of pair-programming partner is raised.
	\par No breaches of academic integrity.}
%
    \criterion{Design of the solution}{15\%}
        \grade\fail No evidence of upfront design
        \grade The correspondence between design and implementation is tenuous.
        \grade The design somewhat corresponds to the final implementation.
        \grade The design corresponds to the final implementation.
        \grade The design clearly corresponds to the implementation.
        \grade The design clearly and comprehensively corresponds to the implementation.
%
    \criterion{Innovation and creative flair}{10\%}
        \grade\fail No evidence of innovation and/or creativity.
        \grade Some evidence of emerging innovation and/or creativity.
            \par The solution is purely derivative of existing products.
            \par There is no evidence of divergent thinking.
        \grade Little evidence of emerging innovation and/or creativity.
            \par The solution is mostly derivative, with some attempts at innovation.
            \par There is evidence of an attempt at divergent thinking.
        \grade Much evidence of emerging innovation and/or creativity.
            \par The solution is an interesting and somewhat innovative product.
            \par There is some evidence of divergent thinking.
        \grade Considerable evidence of mastery of innovative and creative practice.
            \par The solution is a novel and innovative product.
            \par There is much evidence of divergent thinking.
        \grade Significant evidence of mastery of innovative and creative practice.
            \par The solution is a unique and innovative product.
            \par There is significant evidence of divergent thinking.
%            
    \criterion{Functionality of physical prototype}{15\%}
        \grade\fail A physical prototype is not produced, or the prototype is completely non-functional.
        \grade The physical prototype has no functionality.
            \par There are serious technical and/or constructional flaws.
        \grade The physical prototype has some functionality.
            \par There are obvious technical and/or constructional flaws.
        \grade The physical prototype has much functionality.
            \par There are minor technical and/or constructional flaws.
        \grade The physical prototype has considerable functionality.
            \par There are superficial technical and/or constructional flaws.
        \grade The physical prototype has significant functionality.
            \par The technical execution and physical construction are flawless.
%
    \criterion{Sophistication: \par Software \par Electronics \par Physical construction}{15\%}
        \grade\fail The solution lacks even a basic level of sophistication in any of the three areas.
        \grade The solution evidences some sophistication in one or more of the three areas.
            \par Some insight has been demonstrated in any area.
        \grade The solution evidences little sophistication in one or more of the three areas.
            \par Little insight has been demonstrated in at least one of the areas.
        \grade The solution evidences much sophistication in two or more of the three areas.
            \par Much insight has been demonstrated in at least one of the areas.
        \grade The solution evidences considerable sophistication in all three areas.
            \par Considerable insight has been demonstrated in at least two of these areas.
        \grade The solution evidences significant sophistication in all three areas..
            \par Significant insight has been demonstrated in all three areas.
%            
    \criterion{Use of Version Control}{5\%}
        \grade\fail GitHub has not been used.
        \grade Source code has rarely been checked into GitHub.
        \grade Source code  has been checked into GitHub at least once per week.
            \par Commit messages are present.
            \par There is evidence of engagement with peers (e.g.\ code review).
        \grade Source code  has been checked into GitHub several times per week.
            \par Commit messages are clear, concise and relevant.
            \par There is evidence of somewhat meaningful engagement with peers (e.g.\ code review).
        \grade Source code has been checked into GitHub several times per week.
            \par Commit messages are clear, concise and relevant.
            \par There is evidence of meaningful engagement with peers (e.g.\ code review).
        \grade Source code has been checked into GitHub several times per week.
            \par Commit messages are clear, concise and relevant.
            \par There is evidence of effective engagement with peers (e.g.\ code review).
%
\end{markingrubric}

\end{document}