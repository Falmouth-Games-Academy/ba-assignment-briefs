\documentclass{../../fal_assignment}
\graphicspath{ {../../} }

\usepackage{enumitem}
\setlist{nosep} % Make enumerate / itemize lists more closely spaced
\usepackage[T1]{fontenc} % http://tex.stackexchange.com/a/17858
\usepackage{url}
\usepackage{todonotes}

\title{Advanced Programming - Assignment 1}
\author{Brian McDonald}
\module{GAM250}

\begin{document}

\maketitle

\section*{Introduction}

\begin{marginquote}
If we knew what it was we were doing, it would not be called research, would it?

--- Albert Einstein

\end{marginquote}

In this assignment you will be developing your research skills in order to expand your programming knowledge. This research will inform the direction of Assignment 2 and even feed into your 2nd year group project.

Research is a key skill in any programmers toolbox and during your career as a Games Developer you will encounter problems that will require a significant amount of research.

This assignment is formed of several parts:

\begin{enumerate}[label=(\Alph*)]
	\item \textbf{Provide} a project proposal, this should contain the following:
	\begin{enumerate}
		\item Project Title
		\item Project Description
		\item Project Justification
		\item List of key sources  (a minimum of 5, including 3 Journals) 
	\end{enumerate}
    \item A \textbf{draft} of your research
    \item A \textbf{fully referenced document} which contains your research
\end{enumerate}

\subsection*{Part A}

Part A is \textbf{single formative submission}. This is \textbf{individual} work will be assessed on a \textbf{threshold} basis. The following criteria are used to determine a pass or fail: 

\begin{enumerate}[label=(\alph*)]
	\item Submission is timely;
	\item Reasonable justification for the project
	\item At least 5 sources included (minimum of 3 Journals)
\end{enumerate}

To complete Part A, you should submit a document to the assignment area of the learning space by \textbf{Week 3}.

You will receive informal feedback via email from your \textbf{tutor}.


\subsection*{Part B}

Part B is a \textbf{single formal submission}. This is \textbf{individual} work will be assessed on a \textbf{threshold} basis. The following criteria are used to determine a pass or fail: 

\begin{enumerate}[label=(\alph*)]
	\item Submission is timely;
	\item Varied list of references
	\item A reasonable standard of English
	\item Incorporated feedback from Part B
\end{enumerate}

To complete Part B, you should submit a document to the assignment area of the learning space by \textbf{final submission deadline}.

You will receive \textbf{formal} feedback on the learning space

\section*{Additional Guidance}
Research is one of the key skills you can develop as a programmer, during you career you will encounter problems which you can't solve straight away. This will require you to go away and carry out research to gather the knowledge to give you the tools to solve the problem. In addition to this, research is simply not about parroting what has been written before, you have to critically analyse and then adapt to your own use.

At Falmouth University we use the Harvard Referencing style, please ensure that you use this  for all references. You can receive support for this from your tutor and the University Library - \url{http://library.fxplus.ac.uk/library/how/referencing}. You are also welcome to use tools such as RefWorks, Mendely, or Zotero to organise your references, these tools will often allow you to export a reading list in a Harvard style which can then be incorporated into your report.

Writing takes practice so make sure that you take feedback from your tutor and incorporate the changes into a new version of the document. You should also consider getting someone else to proof read your report, it is very easy to make mistakes and not notice them when you are too close to the work.

\section*{FAQ}

\begin{itemize}
	\item 	\textbf{What is the deadline for this assignment?} \\ 
    		Falmouth University policy states that deadlines must only be specified on the MyFalmouth system.
    		
	\item 	\textbf{What should I do to seek help?} \\ 
    		You can email your tutor for informal clarifications. For informal feedback, please consider booking a tutorial slot.
    		
    	\item 	\textbf{Is this a mistake?} \\ 	
    		If you have discovered an issue with the brief itself, the source files are available at: \\
    		\url{https://github.com/Falmouth-Games-Academy/ba-assignment-briefs}.\\
    		 Please raise an issue and comment accordingly.
\end{itemize}

\section*{Additional Resources}

\begin{itemize}
     \item Strunk, W., 2007. The elements of style. Penguin.
     \item Ridley, D., 2012. The literature review: A step-by-step guide for students. Sage.
     \item Harvard Referencing at Falmouth University [Online], Available: \url{http://ask.fxplus.ac.uk/harvard-falmouth} [5 June 2017]
     \item Game Developer Conference Vault [Online], Available: \url{http://www.gdcvault.com/} [5 June 2017]
     \item SIGGRAPH [Online], Available: \url{http://www.siggraph.org/}  [5 June 2017]
     \item DiGRA [Online], Available: \url{http://www.digra.org/}  [5 June 2017]
     \item FDG [Online], Available: \url{https://ispr.info/2016/11/18/call-foundations-of-digital-games-fdg-2017/}  [5 June 2017]
     \item International conference in computational creativity [Online], Available: \url{http://computationalcreativity.net/iccc2017/}  [5 June 2017]
     \item Chi Play [Online], Available: \url{http://chiplay.acm.org/2017/}  [5 June 2017]
     
    
\end{itemize}

\begin{markingrubric}
    \firstcriterion{Basic Competency Threshold}{40\%}
        \gradespan{1}{\fail At least one part is missing or is unsatisfactory. 
        	No attempt to provide references or the references are in the correct format
        
        There is little or no evidence of updates based on feedback from tutor}
        \gradespan{5}{Submission is timely.
	\par Clear evidence of `reasonable' changes based on feedback
	\par Clear evidence of `reasonable' amount of research carried out
	\par At least seven references provided
	\par No breaches of academic integrity.}
%
	\criterion{Choice of Topic}{5\%}
	\grade\fail Topic is not relevant
	\grade Topic is relevant to the field but does not push student's skill
	\grade Topic is relevant and expands on a topic covered in class
	\grade Topic is relevant and challenges the student to conduct `more' interesting research
	\grade Topic is relevant and is very novel
	\grade Topic is relevant and is `ground breaking'
%
%
    \criterion{Quality of writing}{20\%}
        \grade\fail No evidence of analysis of the relevant research in the area
        \grade Some evidence of analysis of the relevant research in the area
        \grade Some evidence of critical analysis of the relevant research in the area
        \grade A good amount of critical analysis of the relevant research in the area
        \grade A very good amount of critical analysis, the student compares and contrasts papers in the relevant area
        \grade An excellent amount of critical analysis, the student is able to construct a good argument based on research
%
    \criterion{Breadth of sources}{10\%}
        \grade\fail All papers from one source.
        \grade An over reliance on web sources, or very little in the way of journal/conference papers 
        \grade Some journal articles provided, most come from one source
        \grade A good varied mix of journal articles and conference papers
        \grade A very good mix of journal articles and conference papers
        \grade An excellent selection of journal articles and conference papers
%            
    \criterion{Report Structure}{15\%}
        \grade\fail The report has no structure or doesn't follow suggested format
        \grade The report uses the suggested structure but has some major flow issues
        \grade The report uses the suggested structure but has some flow issues
        \grade The report uses the suggested structure but has some minor flow issues
        \grade The report uses the suggested structure but has some very minor flow issues
        \grade The report uses the suggested structure and excellent flow
%
%
	\criterion{Standard of English}{10\%}
	\grade\fail The report has many spelling mistakes and grammatical errors	
	\grade The report has spelling mistakes and grammatical errors	
	\grade The report has some spelling mistakes and grammatical errors	
	\grade The report has few spelling mistakes and grammatical errors	
	\grade The report has very few spelling mistakes and grammatical errors	
	\grade The report has no spelling mistakes and grammatical errors	
%

\end{markingrubric}

\end{document}