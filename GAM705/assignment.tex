\chapter{Assignment Structure for the Final Major Project}

\section*{Introduction}

The Final Major project consists of two distinct assignments: the proposal pitch and the final artefact. The first assignment should be used to inform your approach to the second assignment and therefor, should be given significant attention. 

Both assignments can be broken down into a series of interrelated steps: 

\begin{enumerate}[label=(\Alph*)]

	\item \textbf{Conceive} of a topic for your project, by:
		\begin{enumerate}[label=(\roman*)]
			\item \textbf{reviewing} the academic literature \textbf{in addition to} the state-of-the-art
			\item \textbf{and deriving} a key question or problem from the review to motivate your work
		\end{enumerate}

	\item \textbf{Deliver} a 10-minute proposal pitch that will:
		\begin{enumerate}[label=(\roman*)]
			\item \textbf{Utilises} the example proposal structure in manner appropriate to your project
			\item \textbf{explain} the context of your project
			\item \textbf{identify} and \textbf{discuss} the academic literature relevant to your project
			\item \textbf{propose} one or more questions for underpin your project
			\item \textbf{articulate} the ethical considerations you have made
			\item \textbf{illustrate} your approach to project planning and management
		\end{enumerate}
	
	\item \textbf{Formalise} a final proposal that will:
		\begin{enumerate}[label=(\roman*)]
			\item \textbf{address} any issues raised in the proposal pitch
		\end{enumerate}

	\item \textbf{Iterate} across the prototype fidelity continuum:
		\begin{enumerate}[label=(\roman*)]
			\item \textbf{facilitate} the collection of empirical data for your project
			\item \textbf{demonstrate} the technical feasibility of your proposed artefact
			\item \textbf{provide} a basis for further development and experimentation as you progress through the module
		\end{enumerate}
	
	\item \textbf{Complete} your final Major Project artefact, ensuring that you:
		\begin{enumerate}[label=(\roman*)]
			\item \textbf{apply} a rigorous project management approach;
			\item \textbf{follow} best practices in your particular field of specialism;
			\item \textbf{and clearly demonstrate}, where appropriate, validation, verification, testing, and refactoring;
		\end{enumerate}

	\item \textbf{Deliver} a 10-minute presentation that will:
		\begin{enumerate}[label=(\roman*)]
			\item \textbf{showcase} the final artefact;
			\item and \textbf{defend} the work carried out through Q\&A in a viva context.
		\end{enumerate}
\end{enumerate}

\subsection*{Part A}

Part A consists of a \textbf{single formative} submission.

To complete Part A, carry out some preliminary research and experimentation to inform your initial Final Major Project proposal. Lock down a rough research topic or question to drive the project development. Share your initial thoughts with peers and faculty by posting a summary on the associated Learning Space discussion forum. You will receive \textbf{informal feedback}.

This should be done within the first week of the module.

\subsection*{Part B}

Part B consists of a \textbf{single summative submission}.
This work is \textbf{individual} and will be assessed on a \textbf{criterion} basis as defined in the rubric below.
To pass, complete the template format. Each project will have very different requirements so you may need to adapt the template to suit your needs.

\textbf{Suggested Template: }
\begin{enumerate}
	\item title page
	\item overview
	\item user description, including personas
	\item storyboards of user experience
	\item prototypes
	\item features and functionality
	\item the justification for design (theoretical and practical) 
	\item results of preliminary testing 
	\item shortcomings of design
	\item expansion – stretch goals
	\item next steps in the design process
	\item summary
\end{enumerate}

\textbf{Some higher level questions for you to consider: }
\begin{enumerate}[label=(\roman*)]
	\item What is the context of your project? How does it fit into your specific field?
	\item What are the key results from the literature upon which your project will be built?
	\item What is the current state of knowledge in the field. What are the open questions and challenges?
	\item What is (are) the key research question(s) that you will seek to answer in your project?
	\item What are the key legal, social, ethical, and/or professional issues associated with your project?
\end{enumerate}

To complete Part B, prepare a 10 minute presentation and deliver it in the scheduled session. Prepare your slides using your choice of presentation software (e.g.\ Beamer, reveal.js, PowerPoint).

You will receive immediate \textbf{informal feedback} from your tutor and \textbf{formal feedback} within three weeks of the proposal pitch delivery.

\subsection*{Part C}

Part C consists of a \textbf{single formative submission}.

To complete this step, improve upon the proposal document based on the informal feedback from the proposal pitch. You will be expected to provide this document in your first research supervision meeting. 

You will receive immediate \textbf{informal feedback} from your supervisor.

\subsection*{Part D}

Part D consists of a \textbf{multiple formative submissions}.

To complete Part D, carry out an extensive research and development project using industry standard project management principles and techniques. You will receive regular \textbf{informal feedback} about your work through meetings with your supervisor. Iteratively improve the research artefact and show it to your research supervisor in a timetabled meeting. As the requirements for the research artefact will vary by project, consult with your supervisor to verify whether or not the artefact is adequate for the desired purpose.

\subsection*{Part E}

Part E consists of a \textbf{single summative submission} that feeds into part E. 

To complete part E, upload a \texttt{.zip} file containing the final version of your artefact and any assets/dependencies to the LearningSpace. If the files in question are too large for upload, seek advice from your supervisor.

Note that LearningSpace will only accept a single \texttt{.zip} file.

This work is \textbf{individual} and will be assessed on a \textbf{holistic} basis, according to the descriptors set out at the end of this document. You will receive \textbf{formal feedback} three weeks after the summative deadline.

\subsection*{Part F}

Part F consists of a \textbf{single summative submission}.
This work is \textbf{individual} and will be assessed on a \textbf{threshold} basis.
To pass, answer the following questions:

\begin{enumerate}[label=(\roman*)]
	\item What was the purpose of your project?
	\item How did you approach the project?
	\item What did you discover?
	\item What are the implications/value of the work?
\end{enumerate}

To complete Part F, prepare a 10-minute presentation in the timetabled session after the submission deadline.
Prepare your slides using your choice of presentation software (e.g.\ Beamer, reveal.js, PowerPoint).

You will receive immediate \textbf{informal feedback} from tutors.

\section*{FAQ}

\begin{itemize}
	\item 	\textbf{What is the deadline for this assignment?} \\ 
    		Falmouth University policy states that deadlines must only be specified on the MyFalmouth system.
    		
	\item 	\textbf{What should I do to seek help?} \\ 
    		You can email your tutor for informal clarifications. For informal feedback, make a pull request on GitHub. 
    		
    	\item 	\textbf{Is this a mistake?} \\ 	
    		If you have discovered an issue with the brief itself, notify the module leader.
\end{itemize}
