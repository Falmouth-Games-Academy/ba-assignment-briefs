\chapter{Introduction}

You are required to deliver a major \textbf{research project} as part of your Masters degree; in the form of \textbf{empirical} or \textbf{practice-based} research relating to your specialism. Individually, you will explore a field that interests you, and for which there is a clearly identified need.

The higher level aim for this module is to support you on your journey through a research project that has the potential to contribute new knowledge in a chosen field of study, produce an artefact that is of publishable quality, and develop your personal portfolio in a way that supports your future aspirations. The very strongest projects in the past have been submitted and accepted for publication in international conferences or led to further funding and business start-up.

\section*{Selecting a project}

Your \textbf{project supervisor} is responsible for giving you guidance and feedback throughout your project, through regular group and individual supervision meetings.

Every potential supervisor has their own area of expertise and research interests, and has suggested several possible project titles. These are available here:

\url{https://www.falmouth.ac.uk/staff-profiles/?field_department_target_id_selective=34458}

You are required propose your own project title within your field of expertise. One of the few restrictions on the project topic is that it must allow you to develop some form of \textbf{artefact} as this is a component of the assessment. This doesn't have to be a single artefact, it could be a series of experiments or artefacts that form a coherent portfolio. Your supervisor can advise on what would constitute a relevant and suitable artefact for your project.

\section*{Module overview}

For this module, you are required to complete one assignment, composed of two parts: a proposal \textbf{pitch} and an \textbf{artefact}. 

\subsubsection*{GAM705 Final Major Project: Proposal (20\%)}

Your \textbf{proposal pitch} will present your project concept and disseminate your initial research and experimentation. In delivering this proposal pitch you will evidence a familiarity with the wider context of your project, how it relates to the relevant academic literature and the value of the work to be carried out. You are also required to address the ethical issues surrounding your project, and justify your proposed research methods accordingly. A plan should be present that shows you have thought about the time constraints of the module, potential blockers that might hinder development and the milestones that will ensure project delivery. 

You will be provided with a template to help shape the proposal but you \textbf{should} adapt it to suit the requirements of your project. 

\subsubsection*{GAM705: Final Major Project: Artefact (80\%)}

The nature of the artefact will vary dramatically between individuals. However, your work will be assessed on the same shared criteria of viability, design, innovation, functional coherence, quality of defence. This set of criteria have been selected for their transferability and may have slightly different meaning depending on your specialism. For instance, software design usually refers to the planning and implementation of code. Whereas, user experience design will include considerations for the user journey and the perceived usability of an artefect. It is critically important to recognise that in this project you \textbf{must} evidence an iterative process and document your critical thinking. It is your choice how you document your process but tools such as blogs, version control, design portfolios, sketchbooks and many others can be used. When it comes to assessment, it is vital that the assessor is able to identify how your project has developed from initial concept, through a series of prototypes and arrived at the final deliverable.

\subsubsection*{Assessment weighting}

This module constitutes a total of 60 credits and is double the weighting of past modules. In this module, the proposal is worth 20\% of the marks, and the Final Major Project artefact 80\%. 
