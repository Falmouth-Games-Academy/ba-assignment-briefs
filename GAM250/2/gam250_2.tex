\documentclass{../../fal_assignment}
\graphicspath{ {../../} }

\usepackage{enumitem}
\setlist{nosep} % Make enumerate / itemize lists more closely spaced
\usepackage[T1]{fontenc} % http://tex.stackexchange.com/a/17858
\usepackage{url}
\usepackage{todonotes}

\title{Advanced Programming - Assignment 2}
\author{Brian McDonald}
\module{GAM250}

\begin{document}

\maketitle

\section*{Introduction}

\begin{marginquote}
Any code of your own that you haven't looked at for six or more months might as well have been written by someone else.

--- Eagleson's Law

\end{marginquote}

In this assignment, you will work as an individual to develop a game or a resuable component that can be used in a game. This will implementation of the topic you selected for assignment 1. This can be related to your group project or something you want to explore for your own portfolio.

This assignment is formed of two parts:

\begin{enumerate}[label=(\Alph*)]
    \item \textbf{Implement}, a small prototype that shows progress on your topic:
    	\begin{enumerate}[label=\roman*.]
    		\item \textbf{demonstrate} the core idea behind your project
			\item \textbf{contains} some evidence of good software design
		\end{enumerate}
    \item \textbf{Implement} a final version of your application:
    	\begin{enumerate}[label=\roman*.]
    		\item \textbf{Evidence} of optimisation
    		\item \textbf{Demonstrate} your academic integrity;
    		\item \textbf{Show} your programming knowledge \textbf{and} communication skills.
		\end{enumerate}
\end{enumerate}

\subsection*{Assignment Setup}

Fork the GitHub repository at:

\indent \url{https://github.com/Falmouth-Games-Academy/gam250}

Use the existing directory structure and, as required, extend this structure with sub-directories. Ensure that you maintain the \texttt{readme.md} file.

Setup your Unity Project for using version control \url{}

Modify the \texttt{.gitignore} to the defaults for \textbf{Unity}. Please, also ensure that you add Git LFS to your project \url{https://git-lfs.github.com/} and you add editor-specific files and folders to \texttt{.gitignore}.  

\subsection*{Part A}

Part A is a \textbf{single summative submission}. This is \textbf{individual} work will be assessed on a \textbf{threshold} basis. The following criteria are used to determine a pass or fail: 

\begin{enumerate}[label=(\alph*)]
	\item Enough work is available to hold a meaningful discussion; 
	\item Clear evidence of programming knowledge and communication skills; 
	\item No breaches of academic integrity. 
\end{enumerate}

To complete Part A, prepare a practical demonstration. Ensure that the source code and related assets are pushed to GitHub and a pull request is made prior to the scheduled code review session. Then, attend the scheduled code review session. 

You will receive \textbf{immediate informal} feedback from your \textbf{tutor}.

\subsection*{Part B}

Part B is the final submission of the code base to the learning space, this should be a zipped version of your GitHub repository

\begin{enumerate}[label=(\alph*)]
	\item Enough work is available to hold a meaningful discussion; 
	\item Clear evidence of programming knowledge and communication skills; 
	\item No breaches of academic integrity. 
\end{enumerate}

To complete Part B, prepare a practical demonstration. Ensure that the source code and related assets are pushed to GitHub and a pull request is made prior to the scheduled code review session. Then, attend the scheduled code review session. 

\section*{Additional Guidance}
The goal of this module is to build on your experience programming from 1st year and develop your skills in order to create performant and reusable software. You should constantly profile your application and then optimise the most critical path in order to make your application have good performance on most devices. In terms of reusability, you should be able to reuse the systems created in this coursework in any other Unity project including your 2nd and 3rd year group projects.  

\section*{FAQ}

\begin{itemize}
	\item 	\textbf{What is the deadline for this assignment?} \\ 
    		Falmouth University policy states that deadlines must only be specified on the MyFalmouth system.
    		
	\item 	\textbf{What should I do to seek help?} \\ 
    		You can email your tutor for informal clarifications. For informal feedback, make a pull request on GitHub. 
    		
    	\item 	\textbf{Is this a mistake?} \\ 	
    		If you have discovered an issue with the brief itself, the source files are available at: \\
    		\url{https://github.com/Falmouth-Games-Academy/ba-assignment-briefs}.\\
    		 Please raise an issue and comment accordingly.
\end{itemize}

\section*{Additional Resources}

\begin{itemize}
     \item Unity Profiler Overview - \url{https://docs.unity3d.com/Manual/Profiler.html}
     \item Unity Europe 2017 - Performance optimization for beginners -  \url{https://www.youtube.com/watch?v=1e5WY2qf600}
     \item Game Programming Patterns - \url{http://gameprogrammingpatterns.com/contents.html}
\end{itemize}

\begin{markingrubric}
    \firstcriterion{Basic Competency Threshold}{40\%}
        \gradespan{1}{\fail At least one part is missing or is unsatisfactory. 
        
        There is little or no evidence of an iterative development and no improvement over time in regards to the quality of the design}
        \gradespan{5}{Submission is timely.
        	\par Enough work is available to hold a meaningful discussion.
	\par Clear evidence of a `reasonable' iterative development process
	\par Clear evidence of programming knowledge and communication skills.
	\par Clear evidence of reflection on own performance and contribution.
	\par No breaches of academic integrity.}
%
    \criterion{Design of the solution}{15\%}
        \grade\fail No evidence of upfront design
        \grade The correspondence between design and implementation is tenuous.
        \grade The design somewhat corresponds to the final implementation.
        \grade The design corresponds to the final implementation.
        \grade The design clearly corresponds to the implementation.
        \grade The design clearly and comprehensively corresponds to the implementation.
%
    \criterion{Innovation and creative flair}{10\%}
        \grade\fail No evidence of innovation and/or creativity.
        \grade Some evidence of emerging innovation and/or creativity.
            \par The solution is purely derivative of existing products.
            \par There is no evidence of divergent thinking.
        \grade Little evidence of emerging innovation and/or creativity.
            \par The solution is mostly derivative, with some attempts at innovation.
            \par There is evidence of an attempt at divergent thinking.
        \grade Much evidence of emerging innovation and/or creativity.
            \par The solution is an interesting and somewhat innovative product.
            \par There is some evidence of divergent thinking.
        \grade Considerable evidence of mastery of innovative and creative practice.
            \par The solution is a novel and innovative product.
            \par There is much evidence of divergent thinking.
        \grade Significant evidence of mastery of innovative and creative practice.
            \par The solution is a unique and innovative product.
            \par There is significant evidence of divergent thinking.
%            
    \criterion{Profiling}{15\%}
        \grade\fail No profiling is carried out
        \grade Some basic profiling carried out
            \par There is no evidence that profiling results have been used to optimise the game 
        \grade Some basic profiling and optimisation carried out
        \grade Profiling and optimisation carried out throughout the project
        \grade Considerable profiling and optimisation carried out throughout the project
        \grade Significant profiling and optimisation carried out throughout the project
%
    \criterion{Sophisticationn}{15\%}
        \grade\fail The solution lacks even a basic level of sophistication.
        \grade The solution evidences some sophistication
        \grade The solution evidences little sophistication
        \grade The solution evidences much sophistication 
        \grade The solution evidences considerable sophistication
        \grade The solution evidences significant sophistication
%            
    \criterion{Use of Version Control}{5\%}
        \grade\fail GitHub has not been used.
        \grade Source code has rarely been checked into GitHub.
        \grade Source code  has been checked into GitHub at least once per week.
            \par Commit messages are present.
            \par There is evidence of engagement with peers (e.g.\ code review).
        \grade Source code  has been checked into GitHub several times per week.
            \par Commit messages are clear, concise and relevant.
            \par There is evidence of somewhat meaningful engagement with peers (e.g.\ code review).
        \grade Source code has been checked into GitHub several times per week.
            \par Commit messages are clear, concise and relevant.
            \par There is evidence of meaningful engagement with peers (e.g.\ code review).
        \grade Source code has been checked into GitHub several times per week.
            \par Commit messages are clear, concise and relevant.
            \par There is evidence of effective engagement with peers (e.g.\ code review).
%
\end{markingrubric}

\end{document}